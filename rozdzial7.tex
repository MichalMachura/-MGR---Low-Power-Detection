\chapter{Podsumowanie}
\label{cha:Podsumowanie}

% Porównanie rezultatów(narzędzia, kwantyzacje itp.).
% Jeżeli to będzie po ogłoszeniu wyników to wyniki konkursu.

W ramach niniejszej pracy przedstawiono proces projektowania architektury sieci neuronowej z~ zamiarem sprzętowej akceleracji. 
Przedstawiono to na przykładzie zadania detekcji pojedynczego obiektu. 
System ten był opracowywany na potrzeby konkursu \emph{2021 DAC SDC}. 
W tym celu wymagane było dokładne przeanalizowanie wymagań, a~ także zapoznanie z~ docelową platformą -- płytką rozwojową \emph{Avnet Ultra96 V2}. 
Przedstawiono również dostępne narzędzia pozwalające na przejście z~ modelu zmiennoprzecinkowego, przez model kwantyzowany, aż do sprzętowej akceleracji.
Zaproponowanie własnego rozwiązania problemu detekcji wymagało rozeznania się w~ już istniejących. 
W tym celu dokonano przeglądu literatury przedstawiając zarówno rozwiązania klasyczne, jak i~ te wykorzystujące głębokie sieci neuronowe. 
Szczególną uwagę zwrócono na rozwiązania energooszczędne, w~ tym pochodzące z~ poprzednich edycji konkursu.
Uzyskana wiedza pozwoliła poprzez liczne badania na zaproponowanie własnej architektury \emph{LittleNet}. 
Zastosowano tutaj konwwolucje \emph{depthwise} wykorzystującą wielu filtrów dla każdego kanału oraz konwolucje \emph{pointwise}.
Rozwiązanie to osiągało dokładność $IoU = 0.78$ dla modelu zmiennoprzecinkowego.
Przejście przez etap kwantyzacji do zapisu stałoprzecinkowego pozwoliło uzyskać już wartość $IoU = 0.7015$ (przy skalowaniu obrazu z~ wykorzystaniem liczb całkowitych). 
Zdecydowano się na zaprojektowanie własnego akceleratora sprzętowego z~ wykorzystaniem języka \emph{System Verilog}. 
Uzyskany akcelerator po przeprowadzonej optymalizacji osiągał przepustowość rzędu $72.7 fps$ zużywając $2739 J$ energii.
Możliwa jest praca przy częstotliwości nawet $215$ MHz osiągając przepustowość $183 fps$ oraz zużycie energii wynoszące $1070 J$. 
Jednakże wynik ten uzyskiwany był z~ pominięciem odczytu obrazów oraz wszelkiego przetwarzania z~ użyciem systemu procesorowego.
Spowodowane było to niewydajną implementacją programową.

Zaimplementowane rozwiązanie pozwala na osiągnięcie dobrych rezultatów. 
Jednakże, aby rozwiązanie mogło poprawić swój wynik niezbędne jest zaimplementowanie części programowej w~ sposób wydajny np. wykorzystując język \emph{C} wraz z~ przetwarzaniem wielowątkowym.
Zaproponowane typy akceleratorów można poszerzyć m.in. o~ operację pełnej konwolucji.
