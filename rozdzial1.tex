\chapter{Wprowadzenie}
\label{cha:wprowadzenie}

Wstęp...
%---------------------------------------------------------------------------

\section{Cele pracy}
\label{sec:celePracy}
Cele pracy...


Celem pracy jest implementacja wybranej głębokiej sieci neuronowej na wbudowanej platformie obliczeniowej - układzie Zynq UltraScale+ MPSoC. System będzie opracowywany na potrzeby konkursu 2021 DAC System Design Contest. Zadaniem jest detekcja wybranych klas obiektów na zdjęciach zarejestrowanych przez drona. System powinien działać w czasie rzeczywistym, charakteryzować się wysoką dokładności (współczynnik IoU co najmniej 0.7) i przy tym zużywać jak najmniej energii.

%W pierwszym etapie pracy należy dokładnie przeanalizować wymagania konkursu 2021 DAC System Design Contest oraz architektury sieci, które uzyskały wysokie lokaty w poprzednich edycjach. Następnie należy opisać je w ramach przeglądu literatury i uzupełnić o inne, nowe publikacje dotyczące efektywnych energetycznie sieci zrealizowanych w FPGA - kwantyzowanych (QNN), binarnych (BNN) itp. Ponadto należy krótko opisać narzędzia wspomagające projektowanie tego typu sieci (np. Brevitas, FINN, czy Vitis AI) oraz wykorzystywaną platformę sprzętową.

%W drugim etapie pracy należy przeprowadzić badania wstępne tj. sprawdzić skuteczność wybranych wariantów sieci w realizacji postawionego zadania detekcji obiektów. W tym celu należy przeprowadzić szereg eksperymentów symulacyjnych (np. na platformie Google Colaboratory). Na tej podstawie należy dokonać wyboru ostatecznej (lub najlepiej rokującej) architektury.

%W ostatnim etapie należy uruchomić wybraną architekturę na platformie sprzętowej, sprawdzić jej szybkość działania, dokładność (na dostępnym zbiorze testowym) oraz zużycie energii. Następnie należy podjąć próbę optymalizacji ostatniego z parametrów, przy zachowaniu dokładności i szybkości działania.


%---------------------------------------------------------------------------

\section{Zawartość pracy}
\label{sec:zawartoscPracy}

Krótki opis o znajduje się w poszczególnych rozdziałach.
