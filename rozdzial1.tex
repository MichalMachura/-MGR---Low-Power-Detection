\chapter{Wprowadzenie}
\label{cha:wprowadzenie}

Rozwój technologii takich jak pojazdy autonomiczne, roboty humanoidalne, systemy nadzorcze czy systemy automatycznej kontroli wymagają stosowania szeroko rozumianych systemów percepcji o~wysokiej skuteczności.
Możliwe jest to miedzy innymi poprzez realizację zadania detekcji wybranych typów obiektów na podstawie danych z czujników  takich jak kamery, radar czy \emph{LiDAR}.
Otrzymane w rezultacie parametry obiektu, takie jak położenie czy wymiary, pozwalają na podjęcie stosownej akcji np. ominiecie przeszkody czy dotarcie do wybranego celu.
We wspomnianych systemach, często wymagane jest też działanie w~czasie rzeczywistym, przy ograniczonym budżecie energetycznym. 

Rozwiązaniami cechującymi się wysoką skutecznością są algorytmy sztucznej inteligencji, a~w~szczególności sieci neuronowe.
Jednak charakteryzują się on również znaczną złożonością obliczeniową.
W przypadku systemów stacjonarnych problem ten można rozwiązać poprzez zastosowanie urządzeń zapewniających dużą moc obliczeniową, takich jak wydajne karty graficzne (GPU, ang. \emph{Graphics Processing Unit}).
Jednak takie podejście nie zawsze jest możliwe dla zastosowań mobilnych -- pojazdów autonomicznych, robotów, dronów -- gdzie dostępny budżet energetyczny jest ograniczony.
W takich sytuacjach jedną z możliwości jest zastosowanie układów \emph{FPGA} (ang. \emph{field-programmable gate array}) czy \emph{SoC} (ang. \emph{System on Chip})
oraz tzw. sieci kwantyzowanych (ang. \emph{Quantized Neural Networks}).
Rozwiązania te pozwalają osiągać zarówno wysoką skuteczność, niewielkie zużycie energii jak i~wysoką przepustowość pozwalającą na zastosowanie w~systemach czasu rzeczywistego.
Zastosowanie nie ogranicza się jedynie do urządzeń mobilnych. 
Możliwe jest również stosowanie w serwerowniach pozwalając nie tylko przyspieszyć obliczenia, lecz również zredukować zużycie energii.  

Oba wspomniane zagadnienia detekcji oraz akceleracji są na tyle istotne, iż społeczność naukowa organizuje konkursy, takie jak \emph{DAC SDC} (ang. \emph{ 2021 Design automation Conference System Design Contest}) czy \emph{LPVC} (ang. \emph{Low Power Vision Challenge}).
 

\section{Cel pracy}
\label{sec:celePracy}
Celem niniejszej pracy jest implementacja głębokiej sieci neuronowej na wbudowanej platformie obliczeniowej -- układzie \emph{Zynq UltraScale+ MPSoC}.
System jest opracowywany na potrzeby konkursu \emph{2021 DAC SDC} (ang. \emph{ 2021 Design automation Conference System Design Contest}).
Zadaniem jest detekcja obiektów na zdjęciach zarejestrowanych przez drona. 
Wymagane jest osiągnięcie wysokiej przepustowości oraz skuteczności, przy jednoczesnym niewielkim zużyciu energii. 
%Do tego celu należy przeprowadzić niezbędne badania literatury pozwalające na zaproponowanie własnego rozwiązania problemu wykorzystującego akcelerację sprzętową obliczeń.

%---------------------------------------------------------------------------

\section{Zawartość pracy}
\label{sec:zawartoscPracy}
W pracy przedstawiono zagadnienia związane z~uczeniem sieci neuronowych, a~także ich implementacją. 
Z~uwagi na uczestnictwo w~konkursie w~rozdziale \ref{cha:Analiza probemu} przedstawiono szczegółowe wymagania i~założenia dotyczące m.in. implementacji i~oceny rozwiązania. 
Przedstawiono także opis docelowej platformy obliczeniowej oraz omówiono narzędzia wspomagające implementację sprzętową sieci neuronowych.
W rozdziale \ref{ch:detekcja} zaprezentowano przegląd metod detekcji klasycznych oraz opartych o~sieci neuronowe, w~szczególności rozwiązania energooszczędne.
Badania nad architekturą oraz proces uczenia i~kwantyzacji zostały opisane w~rozdziale \ref{cha:Badania wstępne}.
Rozdział \ref{cha:Implementacja} przedstawia implementację opracowanej architektury zarówno programową jak i~sprzętową. 
Uzyskane rezultaty oraz przeprowadzoną optymalizację omówiono w~rozdziale \ref{cha:Optymalizacja}. 
W rozdziale \ref{cha:Podsumowanie} podsumowano pracę oraz przedstawiono wnioski z~niej płynące. 

