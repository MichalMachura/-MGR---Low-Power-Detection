\chapter{Wprowadzenie}
\label{cha:wprowadzenie}

Rozwój technologi takich jak pojazdy autonomiczne, roboty humanoidalne, systemy nadzorcze czy systemy automatycznej kontroli wymagają stosowania rozwiązań o wysokiej skuteczności.
We wspomnianych technologiach, często wymagane jest rozpoznawania otoczenia w czasie rzeczywistym.
Możliwe jest to miedzy innymi poprzez realizację zadania detekcji wybranych typów obiektów za pomocą systemów percepcji takich jak \emph{LiDAR} czy kamery. 

Rozwiązaniami cechującymi się wysoką skutecznością są algorytmy sztucznej inteligencji, a w szczególności sieci neuronowe.
Rozwiązania te jednak cechują się wysoką złożonością obliczeniową.
W przypadku systemów stacjonarnych problem ten można rozwiązać poprzez zastosowanie urządzeń zapewniających dużą moc obliczeniową takich jak karty graficzne.
Rozwiązanie to jednak nie zawsze jest możliwe dla zastosowań mobilnych wymagających technologii o niskim zużyciu energii.
W takich sytuacjach korzystnym wydaje się być zastosowanie układów \emph{FPGA}
oraz tzw. sieci kwantyzowanych (ang. \emph{Quantized Neural Networks}).
Rozwiązania te pozwalają osiągać zarówno wysoką skuteczność, niewielkie zużycie energii jak i wysoką przepustowość pozwalającą na zastosowanie w systemach czasu rzeczywistego.

\section{Cele pracy}
\label{sec:celePracy}
Celem niniejszej pracy jest implementacja głębokiej sieci neuronowej na wbudowanej platformie obliczeniowej - układzie \emph{Zynq UltraScale+ MPSoC}.
System jest opracowywany na potrzeby konkursu \emph{2021 DAC SDC}.
Zadaniem jest detekcja obiektów na zdjęciach zarejestrowanych przez drona. 
Wymagane jest osiągnięcie wysokiej przepustowości oraz skuteczności, przy jednoczesnym niewielkim zużyciu energii. 
Do tego celu należy przeprowadzić niezbędne badania literatury pozwalające na zaproponowanie własnego rozwiązania problemu wykorzystującego akcelerację sprzętową obliczeń.

%---------------------------------------------------------------------------

\section{Zawartość pracy}
\label{sec:zawartoscPracy}
W pracy przedstawiono zagadnienia związane z uczeniem sieci neuronowych, a także ich implementacją. Ze względu na uczestnictwo w konkursie w rozdziale \ref{cha:Analiza probemu} przedstawiono szczegółowe wymagania i założenia dotyczące m.in. implementacji i oceny rozwiązania. Przedstawiono także opis docelowej platformy obliczeniowej oraz omówiono narzędzia wspomagające implementacje sprzętową.
W rozdziale \ref{ch:detekcja} zrealizowano przegląd metod detekcji klasycznych oraz opartych o sieci neuronowe, w szczególności o rozwiązania energooszczędne.
Badania nad architekturą oraz proces uczenia i kwantyzacji zostały opisane w rozdziale \ref{cha:Badania wstępne}.
Rozdział \ref{cha:Implementacja} przedstawia implementację opracowanej architektury zarówno programową jak i sprzętową. 
Uzyskane rezultaty oraz przeprowadzoną optymalizację zawarto w rozdziale \ref{cha:Optymalizacja}. 
W rozdziale \ref{cha:Podsumowanie} podsumowano pracę oraz przedstawiono wnioski z niej płynące. 

